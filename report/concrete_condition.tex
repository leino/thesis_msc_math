\subsection{Outline}

Our final task is to show that all finitely generated projective $k[t_1, \dots, t_n]$-modules are free.

We will not quite accomplish that goal in this section, but we will find a concrete condition formulated in terms of matrices, which implies Quillen-Suslin.
All later sections are then devoted to proving this condition constructively, and thus proving Quillen-Suslin.

The steps taken in this section are the following.

We will first relax the concept of freeness, by introducing the weaker concept of stable freeness.

\begin{definition}
\label{def:stablyfree}
  An $R$ -module $P$ is \emph{stably free} if there are integers $m$ and $n$ such that $P \oplus R^m \cong R^n$.
\end{definition}

To make the following exposition a little less verbose, let us introduce the following notation, which will be used only in this section.

\begin{notation}
  The collection of free $R$-modules is denoted $\free(R)$, and the collection of stably free $R$-modules is denoted $\stablyfree(R)$
\end{notation}

Then we will show that if $M$ is a finitely generated, projective $k[t_1, \dots, t_n]$-module, then $M$ is stably free, that is; $\fgp(k[t_1, \dots, t_n]) \subset \stablyfree(k[t_1,\dots, t_n])$.

The next step is to find a sufficient condition on $R$ such that $\fgp(R) \cap \stablyfree(R) \subset \free(R)$.

Putting the above two conditions together for the case $R = k[t_1, \dots, t_n]$ we will obtain a concrete condition, implying Quillen-Suslin.

\subsection{Characterizing stably free modules}

We now give a characterization of stably free modules.
Note that, since every stably free module is also projective, we are characterizing $\fgp(R) \cap \stablyfree(R)$.

\begin{proposition}
\label{prop:stablyfree_char}
  $ P \in \stablyfree(R)$ iff there is a split short exact sequence of the form
  
  \[
    0 \rightarrow P \rightarrow R^n \rightarrow R^m \rightarrow 0
  \]

  for some $m$, $n$.
\end{proposition}
\begin{proof}
  $P$ is stably free, so there are integers $m$ and $n$ such that $R^n \cong P \oplus R^m$.
  This is equivalent to the above short exact sequence splitting.
  \qed
\end{proof}

We can reinterpret this in more computationally friendly terms.
As we just saw, we have $P \in \stablyfree(R)$ iff $P \cong \ker(f: R^n \surjarrow R^m)$ where $f$ splits i.e. has a right inverse.
In other words, such a $P$ is simply (isomorphic to) the kernel of some right invertible $m \times n$ matrix.

\subsection{On the freeness of stably free modules}

In this section, we characterize those $P \in \stablyfree(R)$ which are also free.
Note that, since stably free modules are projective, we are characterizing free modules in $\fgp(R) \cap \stablyfree(R) = \stablyfree(R)$.

\begin{proposition}
  Suppose $P \cong \ker f$ where $f: R^n \surjarrow R^m$ is some split epimorphism.
  ($\Leftrightarrow P \in \stablyfree(R)$).

  Then $P$ is free iff there is some $r$ and some isomorphism $\hat f$ such that the following diagram commutes

  \begin{displaymath}
    \xymatrix{                         & R^m \oplus R^r \ar[d]^\pi \\
         R^n \ar[r]_f \ar[ur]^{\hat f} & R^m                       \\ }
  \end{displaymath}
\end{proposition}
\begin{proof}
  Suppose $P$ is free. Then $g: P \isomarrow R^r$ for some r.
  Also, since $f$ splits, there is an isomorphism $\varphi: R^m \oplus P \isomarrow R^n$ with $\ker \varphi = 0 \oplus P$.
  Define $\hat \varphi: R^m \oplus P \rightarrow R^m \oplus R^r$ by $\hat \varphi(x,p) = (f \varphi(x,p), g(p) )$.
  Clearly $\pi \hat \varphi = f \varphi$ so that $\pi \hat \varphi \varphi^{-1} = f$, i.e. we can take $\hat f = \hat \varphi \varphi^{-1}$.
  Suppose instead that $\hat f$ and $r$ exists, such that the above diagram commutes.
  We have $f = \pi \hat f$, so $P \cong \ker f = \ker (\pi \hat f) \cong \ker \pi = 0 \oplus R^r \cong R^r$.
  \qed
\end{proof}

We can interpret this in terms of matrices, as follows.

We have seen that $P \in \fgp(R) \cap \stablyfree(R)$ is simply the kernel of some right invertible $m \times n$ matrix $M_f$.
Then the condition for freeness presented above is that there is an \emph{invertible} $(m+r) \times n$ matrix $M_{\hat f}$ such that $M_f = M_\pi M_{\hat f}$.
Here the $m \times (m+r)$ matrix $M_\pi$ is the matrix projecting onto the first $m$ coordinates.

Thus $M_f = M_\pi M_{\hat f}$ simply means that we can take the top $m$ rows of the invertible matrix $M_{\hat f}$ and obtain our original matrix $M_f$.

However, it is better to think of it as saying that we can add rows onto $M_f$ until we get an invertible matrix $M_{\hat f}$.

We have just shown the following interpretation.

\begin{proposition}
  Suppose that $P \in \fgp(R) \cap \stablyfree(R)$, so that $P$ is isomorphic to the kernel of some right-invertible matrix $M$.
  Then $P$ is free iff we can obtain an invertible matrix by adding rows to $M$.
\end{proposition}

The above characterization only tells us if a single $P \in \fgp(R) \cap \stablyfree(R)$ is free.

We are ultimately interested in a condition that says that \emph{every} $P \in \fgp(R) \cap \stablyfree(R)$ is free, i.e. to show $\fgp(R) \cap \stablyfree(R) \subset \free(R)$.

A \emph{necessary} condition for this to be the case is that, in particular, any right invertible row can be completed to an invertible matrix by adding rows.

This is also a \emph{sufficient} condition, since a matrix being right invertible implies that each of its rows must be right invertible.

We have just shown the following theorem.

\begin{theorem}
\label{thm:quillensuslin_ringcondition}
  $\fgp(R) \cap \stablyfree(R) \subset \free(R)$ iff every right invertible row vector with entries in $R$ can be completed to an invertible matrix.
\end{theorem}

In the next section, we will show that $\fgp(k[t_1, \dots, t_n]) \subset \stablyfree(k[t_1, \dots, t_n])$.
Then the truth of the above theorem for $R=k[t_1,\dots,t_n]$ will imply Quillen-Suslin.

Finally, we introduce a synonym for right invertible, which is common in the literature.

\begin{definition}
  A right invertible row vector over $R$ is called a \emph{unimodular row over $R$}.
  The set of all such rows is denoted $\Um_n(R)$.
\end{definition}

\subsection{Finitely generated, projective $k[t_1, \dots, t_n]$-modules are stably free}

The statement in the title is almost the Quillen-Suslin theorem.
In light of theorem \ref{thm:quillensuslin_ringcondition}, the truth of this statement gives us a way to prove Quillen-Suslin in a very concrete way.

We will see that the result follows from Hilbert's syzygy theorem.

\begin{theorem}[Hilbert's syzygy theorem]
\label{thm:hilbertsyzygy}
  Let $M \in \fg(k[t_1, \dots, t_n])$.
  Then there is an exact sequence

  \[
    0 \rightarrow F_n \rightarrow \dots \rightarrow F_0 \rightarrow M \rightarrow 0
  \]

  where the $F_i$ are all free and finitely generated.
\end{theorem}

For a proof, see for instance \citep[p.~208]{kunz85}.

The exact sequence in Hilbert's syzygy theorem is called a \emph{finite free resolution} for $M$.

\begin{proposition}
\label{thm:fgppolyringfree}
  Let $M$ be a projective module with a finite free resolution. Then $M$ is stably free.
\end{proposition}
\begin{proof}
  We have
  \[
    0 \rightarrow F_n \overset{\phi_n} \rightarrow \dots \rightarrow F_0 \overset{\phi_0}\rightarrow M \rightarrow 0
  \]
  where each $F_i$ is free.
  We will proceed by induction on $n$.
  For $n = 0$ we are done, since in this case $M \cong F_0$ so that $M$ is free.
  Now, consider $n > 0$.
  Denoting $M_1 = F_1 / \ker{\phi_1}$, we have the following short exact sequence
  \[
    0 \rightarrow M_1 \overset{\overline{\phi_1}}\rightarrow F_0 \overset{\phi_0}\rightarrow M \rightarrow 0
  \]

  Since $M$ is projective we see by proposition \ref{prop:projchar} that $\phi_0$ splits.
  Therefore $F_0 = M_1 \oplus M$. Thus we are done if we can show that $M_1$ is stably free.
  But since $M_1$ is a summand of a free module, we see that $M_1$ is projective.
  We also have the following exact sequence

  \[
    0 \rightarrow F_n \overset{\phi_n} \rightarrow \dots F_2 \overset{\phi_2}\rightarrow F_1 \rightarrow M_1 \rightarrow 0
  \]

  where again the last module, $M_1$, is projective.

  By the induction hypothesis, we see that $M_1$ is stably free and so we are done.
  \qed
\end{proof}

By Hilbert's syzygy theorem and the above proposition, we have trivially the following result.

\begin{corollary}
\label{cor:fgppolyringstablyfree}
  $\fgp(k[t_1, \dots, t_n]) \subset \stablyfree(k[t_1, \dots, t_n])$
\end{corollary}

Using theorem \ref{thm:quillensuslin_ringcondition} and the above corollary, we obtain.

\begin{corollary}\label{cor:quillensuslin_reduced}
  Quillen-Suslin is true iff every unimodular row with entries in $k[t_1,\dots,t_n]$ can be completed
  to an invertible matrix.
\end{corollary}

\subsection{A slightly different condition}
\label{sec:a_slightly_different_condition}

In the previous section, we reduced the proof of the Quillen-Suslin theorem to the
problem of showing that any unimodular row over $k[t_1, \dots, t_n]$ can be completed to an invertible matrix (corollary \ref{cor:quillensuslin_reduced}).

We will now reformulate this condition slightly, in preparation of the following sections.

\begin{notation}
  Let $\alpha,\beta \in R^n$ for some $n$.
  Then we write $\alpha \sim \beta$ if there is $M \in \GL_n(R)$ such that $\alpha M = \beta$.
  We write $\alpha \sim_G \beta$ if there is $M \in G$ such that $\alpha M = \beta$, where $G$ is some
  subgroup of $\GL_n(R)$.
\end{notation}

\begin{proposition}
$\alpha \sim_G (1,0,\dots,0)$ iff $\alpha$ can be completed to an element of $G$.
\end{proposition}

\begin{proof}

\[
  \Rightarrow
\]

Suppose $\alpha M = (1,0,\dots,0)$.
Write
\[
  M^{-1}
  =
  \left(
  \begin{array}{c}
    \beta_1 \\
    \vdots     \\
    \beta_n \\
  \end{array}
  \right)
\]

Where the $\beta_1, \dots, \beta_n$ are rows of length $n$.
Then let

\[
  M_\alpha
  =
  \left(
  \begin{array}{c}
    \alpha  \\
    \beta_2 \\
    \vdots  \\
    \beta_n \\
  \end{array}
  \right)
\]

Then $M_\alpha M = \mathbb{I}_n$ and thus by uniqueness of inverse, we have $M_\alpha = M^{-1} \in G$.

\[
  \Leftarrow
\]

Using similar notation as above, suppose that

\[
  M_\alpha
  =
  \left(
  \begin{array}{c}
    \alpha  \\
    \beta_2 \\
    \vdots  \\
    \beta_n \\
  \end{array}
  \right)
  \in
  G
\]

There is some $M_\alpha^{-1} \in G$ such that $M_\alpha M_\alpha^{-1} = \mathbb{I}_n$.
This implies $\alpha M_\alpha^{-1} = (1,0,\dots,0)$.
  \qed
\end{proof}


\begin{remark}
  In light of corollary \ref{cor:quillensuslin_reduced} and the above proposition, we see that
  the particular $G$ does not really matter since corollary \ref{cor:quillensuslin_reduced} states that the matrix by which our $\alpha \in \Um_n(R)$ is conjugate to $(1,0,\dots,0)$ need only be \emph{invertible},
  i.e. an element of $\GL_n(R)$. Therefore we often write $\sim$ in place of $\sim_G$, and only remark on $G$ when we
  find it worthwhile or interesting.
\end{remark}

For clarity and future reference, we now combine corollary \ref{cor:quillensuslin_reduced}, and the above proposition.

\begin{corollary}
\label{cor:quillen_suslin_concrete}
  Quillen-Suslin is true iff for any $f \in \Um_n(k[t_1, \dots, t_m])$ we have $f \sim (1,0,\dots,0)$.
\end{corollary}

This is the most important result of this section.

To summarize, a constructive proof of Quillen-Suslin amounts to an algorithm which, given a unimodular row $f$ over $k[t_1, \dots, t_m]$, produces an invertible matrix $G \in \GL_n(k[t_1, \dots, t_m])$ such that $fG = (1,0,\dots,0)$.

