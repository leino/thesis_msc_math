In this section, we prove the Quillen-Suslin theorem in the case where $n=1$ so that the base ring $R=k[t]$ is a euclidean domain.
The proof basically reduces to the euclidean algorithm.
This result will be used to give the general proof later on.

\begin{proposition}
\label{prop:generaleuclid}
  Let $R$ be a euclidean domain and let $\alpha = (a_1, \dots, a_n) \in R^n$.
  Then $\alpha \sim_G (g,0,\dots,0)$ where $G = \EL_n(R)$ and $g = \gcd(a_1, \dots, a_n)$.
\end{proposition}
\begin{proof}
  We can apply the ordinary euclidean algorithm to $(a_1, \dots, a_n)$, which is just a sequence
  of elementary transformations transforming $(a_1, \dots, a_n)$ into $(g,0,\dots,0)$.
  The product of these elementary transformations is an element in $\Gamma \in \EL_n(R)$ such
  that $(a_1, \dots, a_n) \Gamma = (g, 0 \dots, 0)$.
  \qed
\end{proof}

\begin{corollary}
\label{cor:quillensuslin_euclidean}
  Let $R$ be a euclidean domain and let $\alpha = (a_1, \dots, a_n) \in \Um_n(R)$.
  Then $\alpha \sim_G (1,0,\dots,0)$ where $G = \EL_n(R)$.
\end{corollary}
\begin{proof}
  The fact that $\alpha$ is unimodular implies that $\gcd(a_1, \dots, a_n)$ is a unit.
  Thus, by proposition $\ref{prop:generaleuclid}$, there is $\Gamma \in \EL_n(R)$ such that
  $\alpha \Gamma = (1, 0, \dots, 0)$.
  \qed
\end{proof}

By corollary \ref{cor:quillensuslin_reduced}, this implies Quillen-Suslin for the case $n=1$.
