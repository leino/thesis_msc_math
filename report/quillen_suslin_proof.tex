\subsection{Outline}
In this section we will put together a proof of the Quillen-Suslin theorem.
More specifically, we will prove the condition given in corollary \ref{cor:quillen_suslin_concrete} for the particular case of $R = k[t_1, \dots, t_m]$.
Namely that if we have a
unimodular row $f = (f_1, \dots, f_n) \in \Um_n(k[t_1, \dots, t_m])$ then $f \sim (1,0,\dots,0)$.
By the mentioned corollary, this implies Quillen-Suslin.

Note that the following is a constructive argument, assuming Suslin's lemma.
Suslin's Lemma will be proved in the sections following this one.

\subsection{Proving Quillen-Suslin using Suslin's Lemma}

We first state Suslin's Lemma.

\begin{lemma}[Suslin's Lemma]\label{lemma:suslinslemma}
  Let $R$ be a commutative ring and suppose $f \in \Um_n(R[t])$. 
  Suppose further that either
  \begin{itemize}
    \item $n = 1,2$
    \item $n \geq 3$ and $f$ has a monic component.
  \end{itemize}
  Then $f(t) \sim f(0)$.
\end{lemma}

As noted above, we will give the proof later, of course \emph{without using any of the results we derive from it here}.


\begin{lemma}[Nagata]
\label{lemma:nagata}
  Let $f \in k[t_1,\dots,t_m]$.
  Then there is an automorphism $\varphi: k[t_1,\dots,t_m] \rightarrow k[t_1,\dots,t_m]$ such that
  $\varphi(f)$ is monic with respect to $t_1$.
\end{lemma}

This is a standard result.
For instance, see \citep[p. ~103]{lam06} for a constructive proof.

% the proof is commented out
\begin{comment}
\begin{proof}
  We construct the automorphism explicitly by changing the variables.
  We denote our initial guess by $\psi$:

  \[
    t_1 \xrightarrow[]{\psi} t_1
  \]
  \[
    t_i \xrightarrow[]{\psi} t_i + t_1^{r_i}
  \]

  This is clearly an automorphism.
  Now, $\psi$ acts on $f$ in the following way

  If we denote

  \[
    f(t_1, \dots, t_m) = 
    \sum_{I \in S}
    a_{i_1, \dots, i_d}
    t_1^{i_1} t_2^{i_2} \dots t_d^{i_d}
  \]

  (where $I$ are the d-tuples $(i_1, \dots, i_d)$ and $S$ is such that $a_I \neq 0$ for all $I \in S$.)

  we get

  \[
    \psi f(t_1, \dots, t_d) =
    f(t_1, t_2 + t_1^{r_2}, \dots, t_d + t_1^{r_d})
    =
    \sum_{i_1, \dots, i_d \geq 0}    
    a_{i_1, \dots, i_d}
    t_1^{i_1} (t_2 + t_1^{r_2})^{i_2} \dots (t_d + t_1^{r_d})^{i_d}
    =
  \]
  \[
    =
    \sum_{i_1, \dots, i_d \geq 0}    
    a_{i_1, \dots, i_d}
    t_1^{i_1}
    (t_1^{r_2 i_2 + \dots + r_d i_d} + h_{i_1, \dots, i_d})
  \]

  where the $h_{i_1, \dots, i_d}$ are polynomials with $\degree_{t_1} h_{i_1, \dots, i_d} < \sum_{j=2}^d r_j d_j$.
  Therefore

  \[
    \psi f(t_1, \dots, t_d)
    =
    \sum_{i_1, \dots, i_d \geq 0}    
    a_{i_1, \dots, i_d}
    t_1^{i_1 + r_2 i_2 + \dots + r_d i_d}
    +
    h
  \]

  where the $t_1$-degree of $h$ is strictly smaller than that of the sum.
  Assuming that the sum is not zero, we are done if we set $\varphi = \psi / \lc_{t_1}(\psi f_1)$.
  To ensure this, we simply pick the $r_2, \dots, r_j$ so that the numbers $i_1 + r_2 i_2 + \dots + r_d i_d$ are all distinct.
  This is easy to do: choose $m > \max(i_j | i \in S, 1 \leq j \leq d)$, which is defined since $S$ is finite.
  Then we may take $r_j = m^j$. Then the $i_1 + r_2 i_2 + \dots + r_d i_d$ are simply expansions of distinct
  $d$-tuples $(i_1, \dots, i_d)$ ($d$-digit numbers) in an $m$-ary number system, and are therefore distinct.
  \qed
\end{proof}
 \end{comment}

%%%%%%%%%%%%%%%%%%%%%%%%%%%%



By Nagata and Suslin's lemma, we have the following corollary.

\begin{corollary}\label{cor:suslinscorollary}
  Let $R = k[t_1, \dots, t_m]$ and $f \in \Um_n(R[t])$.
  Then $f(t) \sim f(0)$
\end{corollary}

\begin{theorem}
  Let $f \in \Um_n(k[t_1, \dots, t_m])$. Then $f \sim (1,0,\dots,0)$.
\end{theorem}
\begin{proof}
  We proceed by induction on $m$.
  The base case, $m=1$, has already been taken care of.
  (This is the case of a euclidean domain, and so we are done by corollary \ref{cor:quillensuslin_euclidean}.)

  To take the induction step, we use the above corollary.
  \qed
\end{proof}

We have now proved Quillen-Suslin using Suslin's Lemma.
The remainder of this paper is devoted to proving Suslin's Lemma.
